%
% File report.tex
%
% Contact: koller@ling.uni-potsdam.de, yusuke@nii.ac.jp
%%
%% Based on the style files for ACL-2013, which were, in turn,
%% Based on the style files for ACL-2012, which were, in turn,
%% based on the style files for ACL-2011, which were, in turn, 
%% based on the style files for ACL-2010, which were, in turn, 
%% based on the style files for ACL-IJCNLP-2009, which were, in turn,
%% based on the style files for EACL-2009 and IJCNLP-2008...

%% Based on the style files for EACL 2006 by 
%%e.agirre@ehu.es or Sergi.Balari@uab.es
%% and that of ACL 08 by Joakim Nivre and Noah Smith

\documentclass[11pt]{article}
\usepackage{report}
\usepackage{times}
\usepackage{url}
\usepackage{latexsym}
\usepackage{listings}
\usepackage{amsmath}
\usepackage{float}
\usepackage{xcolor}
\usepackage{textcomp}
\usepackage{graphicx}
\usepackage{enumitem}
\restylefloat{table}

\lstset{
  language=C,                % choose the language of the code
  backgroundcolor=\color{white},  % choose the background color. You must add \usepackage{color}
  basicstyle=\ttfamily\scriptsize,
  showspaces=false,               % show spaces adding particular underscores
  showstringspaces=false,         % underline spaces within strings
  showtabs=false,                 % show tabs within strings adding particular underscores
  tabsize=2,                      % sets default tabsize to 2 spaces
  captionpos=b,                   % sets the caption-position to bottom
  breaklines=true,                % sets automatic line breaking
  breakatwhitespace=true,         % sets if automatic breaks should only happen at whitespace
}

%\setlength\titlebox{5cm}

% You can expand the titlebox if you need extra space
% to show all the authors. Please do not make the titlebox
% smaller than 5cm (the original size); we will check this
% in the camera-ready version and ask you to change it back.


\title{CS267 Assignment 2: Parallelize Particle Simulation}

\author{Hang Su \\
  Dept. of EECS, UC Berkeley \\
  International Computer Science Institute \\
  {\tt suhang3240@gmail.com}
  \And
  Andreas Borgen Longva \\
  Dept. of Mathematics, UC Berkeley \\
  NTNU \\
  {\tt andreas.b.longva@gmail.com}
}
\date{}

\begin{document}
\maketitle
\vspace{-8mm}
\begin{abstract}
This report investigates the parallelization of Particle Simulation using different parallelization models. 
We parallelize particle simulations using OpenMP, Pthreads and MPI, and compare their scaling performances.
We show that these techniques can effectively speed up the simulation, achieving an 40\% average scaling efficiency.
percentage of Peak could be achieved.
\end{abstract}

\section{Introduction}
Particle simulation are used in mechanics, biology, astronomy, etc. In this assignment, we will parallelize 
a toy particle simulator.

Naive particle simulation requires a complexity of $O(n^2)$ as forces between every pair of particles are computed.
However, if appropriate approximation is applied, we could reduce the complexity to $O(n)$. The approximation lies in
that only nearby particles have big influences on a particle, so we could divide the spaces into bins, and 
update status of each particles concerning only the interaction between nearby bins.

\begin{table}[htb]
  \centering
  \begin{tabular}{l}
    \hline
      Naive particle simulation\\
    \hline
      \lstinputlisting{samples/naive.cpp}\\
    \hline
  \end{tabular}
  \label{tab:naive}
\end{table}

\begin{table}[htb]
  \centering
  \begin{tabular}{l}
    \hline
      Binned particle simulation\\
    \hline
      \lstinputlisting{samples/serial.cpp}\\
    \hline
  \end{tabular}
  \label{tab:serial}
\end{table}

\section{Parallelization}
\subsection{Shared Memory Models}
OpenMP and Pthreads are two progamming models for shared memory parallelization. We implement both of them and try to 
compare the performance.

\begin{table}[htb]
  \centering
  \begin{tabular}{l}
    \hline
      Shared Memory Parallelization\\
    \hline
      \lstinputlisting{samples/shared_memory.cpp}\\
    \hline
  \end{tabular}
  \label{tab:shared_memory}
\end{table}

\subsection{Distributed Memory Model}
We use MPI \cite{mpi} as parallelization model in this section. An basic approach would be describes as

\begin{table}[htb]
  \centering
  \begin{tabular}{l}
    \hline
      MPI Parallelization\\
    \hline
      \lstinputlisting{samples/mpi_naive.cpp}\\
    \hline
  \end{tabular}
  \label{tab:mpi_naive}
\end{table}

\section{Experiments}
\subsection{Serial Implementation}

\subsection{Shared Memory Models}

\subsection{Distributed Memory Models}

\subsection{Discussions}

\section{Conclusion}


% include your own bib file like this:
\bibliographystyle{acl}
\bibliography{report}

\end{document}
